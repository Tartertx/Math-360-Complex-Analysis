\documentclass[12 pt]{article}        	%sets the font to 12 pt and says this is an article (as opposed to book or other documents)
\usepackage{amsfonts, amssymb}
\usepackage{amsmath}
\usepackage{graphicx}
\usepackage{esint}
\usepackage{float}

\usepackage[left=2cm,right=2.5cm,top=2cm,bottom=1.5cm]{geometry}

\pagestyle{myheadings}

\newcommand{\eqn}[0]{\begin{array}{rcl}}%begin an aligned equation - allows for aligning = or inequalities.  Always use with $$ $$
\newcommand{\eqnend}[0]{\end{array} }  	%end the aligned equation

\newcommand{\qed}[0]{$\square$}

% personalized commands
\newcommand{\inv}{^{-1}}
\newcommand{\bij}{\mathrlap{\hookrightarrow}\rightarrow}
\newcommand{\onto}{\twoheadrightarrow}
\newcommand{\z}{\mathbb Z}
\newcommand{\real}{\mathbb R}
\newcommand{\q}{\mathbb Q}
\newcommand{\complex}{\mathbb C}
\newcommand{\n}{\mathbb N}
\newcommand{\cont}{\lightning}
\newcommand{\osub}{\stackrel{\circ}{\subset}}


\usepackage{amsthm} % Recommended for theorem-like environments
\newtheorem{thm}{Theorem}
\newtheorem{definition}{Definition}
\newtheorem{prop}{Proposition}
\newtheorem{ex}{Example}

\title{Math 360 Homework 3}
\begin{document}
\maketitle
\begin{center}
    Timothy Tarter

    James Madison University

    Department of Mathematics
\end{center}

\section*{Problem 1: Find a value of z for which $1^z \neq 1$.}
Note that we must not use the Principal log, otherwise $1^z \cong 1$. Thus, recall that 
\begin{equation}
    log(z) = log_e|z|+i(Arg(z)+2k\pi)
\end{equation}
and
\begin{equation}
    z^c = exp(clog(z)).
\end{equation}
Then
\begin{align}
    (1+0i)^c &= exp(c[log_e|1|+i(Arg(1+0i)+2k\pi)])\\
    &=exp(ic(Arg(1+0i)+2k\pi))\\
    &=e^{ic(0+2k\pi)}.
\end{align}
Then
\begin{equation}
    2cik\pi \neq 0.
\end{equation}
So letting $c=a+bi$,
\begin{equation}
    (a+bi)(2i\pi k) = -2b\pi k+i(2a\pi k).
\end{equation}
(7) is nonzero for any nonzero $b$ and $a \in \z$. So any $z = a+bi$ where $a\in \z$ and $b \neq 0 \subseteq \real$ will work. 


\section*{Problem 2: Explain why $sin(z)$ and $cos(z)$ are unbounded as complex functions.}
Recall that 
\begin{equation}
    cos(z) = \frac{e^{iz}+e^{-iz}}{2}= \frac{e^{-b+ai}+e^{b-ai}}{2}=\frac{1}{2}\left[ e^{-b}(cos(a)+isin(a)) + e^b (cos(a) - isin(a)) \right]
\end{equation}
and
\begin{equation}
    sin(z) = \frac{e^{iz}-e^{-iz}}{2i} = \frac{e^{-b+ai}-e^{b-ai}}{2i}=  \frac{1}{2i}\left[ e^{-b}(cos(a)+isin(a)) - e^b (cos(a) - isin(a)) \right].
\end{equation}

$\newline$
Then we can take the limit as $b \to \pm\infty$ and get:
\begin{align}
    \lim_{b\to \infty}\frac{1}{2}\left[ e^{-b}(cos(a)+isin(a)) + e^b (cos(a) - isin(a)) \right] = \frac{1}{2}e^\infty(cos(a)-isin(a)) =\infty \\
    \lim_{b\to \infty} \frac{1}{2}\left[ e^{-b}(cos(a)+isin(a)) + e^b (cos(a) - isin(a)) \right]= \frac{1}{2}e^\infty (cos(a)+isin(a)) = \infty\\
    \lim_{b\to -\infty} \frac{1}{2i}\left[ e^{-b}(cos(a)+isin(a)) - e^b (cos(a) - isin(a)) \right] = \frac{1}{2i}e^\infty (cos(a)+isin(a)) = \infty\\
    \lim_{b\to -\infty} \frac{1}{2i}\left[ e^{-b}(cos(a)+isin(a)) - e^b (cos(a) - isin(a)) \right] = \frac{1}{2i}e^\infty(cos(a)-isin(a)) = \infty.
\end{align}
So $sin(z)$ and $cos(z)$ are unbounded complex functions.



\section*{Problem 3: Prove that $cos(2z) = cos^2(z) - sin^2(z)$ holds for any $z\in \complex$.}
Recall equations (8) and (9) which say that 
\begin{equation}
    cos(z) = \frac{e^{iz}+e^{-iz}}{2}
\end{equation}
and
\begin{equation}
    sin(z) = \frac{e^{iz}-e^{-iz}}{2i}.
\end{equation}
It follows then that
\begin{align}
    cos^2(z) &= \left[\frac{e^{iz}+e^{-iz}}{2}\right]^2 \\
    &=\frac{1}{4}[e^{2iz}+e^{-2iz}+2e^{iz-iz}]\\
    &=\frac{1}{4}[e^{2iz}+e^{-2iz}+2]
\end{align}
and
\begin{align}
    sin^2(z) &= \left[\frac{e^{iz}-e^{-iz}}{2i}\right]^2\\
    &= -\frac{1}{4}[e^{2iz}+e^{-2iz}-2e^{iz-iz}]\\
    &= -\frac{1}{4}[e^{2iz}+e^{-2iz}-2]
\end{align}
Then, 
\begin{align}
    cos^2(z) - sin^2(z) &= \frac{1}{4}[([e^{2iz}+e^{-2iz}+2])-(-[e^{2iz}+e^{-2iz}-2])]\\
    &=\frac{1}{4}[2e^{2iz}+2e^{-2iz}+2-2]\\
    &= \frac{e^{2iz}+e^{-2iz}}{2} \\
    &= cos(2z)
\end{align}
as desired.

\hfill\qed


\section*{Problem 4: Find all values of $z$ which satisfy $Log(z) = \frac{i\pi}{4}$.}
Recall that 
\begin{equation}
    Log(z) = ln|z|+iArg(z).
\end{equation}
Then
\begin{align}
    ln|z| = 0 &\Rightarrow |z| = 1\\
    &\text{and}\\
    iArg(z) = i\frac{\pi}{4} &\Rightarrow Arg(z) = \frac{\pi}{4}.
\end{align}
Notably, the only point in $\complex$ with magnitude $1$ and Argument $\frac{\pi}{4}$ is $\frac{\sqrt{2}}{2}+\frac{\sqrt{2}}{2}i$. So $z = \frac{\sqrt{2}}{2}+\frac{\sqrt{2}}{2}i$.



\end{document}